\documentclass{ltjsarticle}
\usepackage{luatexja}
\usepackage{luacode}
\usepackage{amsmath}

\begin{luacode*}
    function my_random(n)
        tex.sprint(math.random(n))
    end
\end{luacode*}
\newcommand{\myRandom}[1]{%
    \directlua{my_random(#1)}%
}

\title{Luaで遊んでみよう}
\author{はんちゃん}
\date{\today}

\begin{document}
\maketitle
$\sqrt{2}$ は \directlua{tex.sprint(math.sqrt(2))} です。しかし \verb|\directlua{}| の内部は lua 言語で書くはずが、\TeX 言語の仕様に準拠してしまうという欠点があります。\\
そこでプリアンブルに \verb|\usepackage{luacode}| を追加して、\verb|luacode*| 環境を使います。\\
早速使ってみましょう。\\
42 mod 5 の答えは
\begin{luacode*}
    local value = 42 % 5
    tex.sprint(value)
\end{luacode*}
です。

\verb|\begin{}\end{}|を使って書くのですが、デフォルトではブロック形式にならず、inlineに表示されます。ちなみに \verb|luacode*| と \verb|luacode| との違いは微妙にあるようですが、ほとんどの場合 \verb|luacode*| としていれば問題ないようです。ただし、 \verb|luacode*| ブロックの中ではマクロが使えない点は注意です。
\verb|luacode| ブロックでは使えます。

\section{マクロを書いて表示させてみよう}

プリアンブルらへんに関数とマクロの定義をして、マクロを呼び出すだけで使えます。\\
$\myRandom{3}, \myRandom{8}, \myRandom{10}$

\section{FizzBuzz}
\begin{luacode*}
    function fizz_buzz()
\end{luacode*}


\end{document}