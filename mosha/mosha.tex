\documentclass{ltjsarticle}
\usepackage{luatexja}
\usepackage{amsmath}

\title{応用数学II(第1回)}
\date{2020年4月14日}
\begin{document}
\maketitle
\section*{課題1}
 (p.4, 問題 1-1,3) 直線 $x + y = −1$ は微分方程式 (1.6):
$$
    \frac{dy}{dx}=x+y
$$
の 1 つの特解であることを確かめよ(ヒント:$x+y=-1$が(1.6)の十分条件であることを示せばよい. すなわち, (1.6) の両辺をそれぞれ計算し同じ値になればよい).
\section*{課題2}
 (p.7, 問題 1-3,1) 微分方程式 (1.9):$dy/dx = f (x) $の$ f (x) $が以下の場合の一般解を求めよ(注意:積 分定数を忘れないこと.以下の問題でも同様.).\\
(c) $(1+x)^{1/2}$
\section*{課題3}
 (p.7, 問題 1-3,2) 微分方程式 (1.9) の点 $(x, y) = (x_0, y_0)$ を通る特解は
$$
    y=\int_{x_0}^xf(x)dx+y_0
$$

\begin{align}
    y=\int_{x_0}^xf(x)dx+y_0
\end{align}
であることを証明せよ.
\section*{課題4}
 (p.11, 問題 1-4,1) 微分方程式 (1.10):

\section*{課題5}
\section*{課題6}
\end{document}